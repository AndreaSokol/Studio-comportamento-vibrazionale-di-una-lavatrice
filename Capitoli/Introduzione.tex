\section{Introduzione}
L’elaborato di seguito esposto è frutto di ricerche condotte attraverso diverse fonti. 
Grazie al materiale didattico gentilmente offerto dal prof. Marco Amabili, libri di testo ed alcune informazioni di cui si è usufruito online, dopo un’attenta analisi sulla veridicità e provenienza, si è riusciti a stilare la presente relazione incentrata sullo studio delle dinamiche facenti parte di un preciso sistema meccanico: la lavatrice.
\\
\\
Il corso di Meccanica delle Vibrazioni ha come obiettivo lo studio di sistemi meccanici soggetti al fenomeno, a volte sottovalutato, delle vibrazioni.

Alcuni degli inconvenienti avvenuti nel corso del tempo sono stati conseguenza di un’inosservanza da parte del personale tecnico su una questione di fondamentale importanza nell’ingegnerizzazione di alcuni prototipi.
Tra questi vi è appunto la lavabiancheria, la quale è stata frutto di anni di studio e miglioramenti attuati nel corso del tempo per limitare quanto più possibile le fastidiose vibrazioni che, oltre che influire acusticamente, tendono a non ottimizzare la dinamica del sistema.
\\
\\
Il presente documento si prefigge lo studio di uno schema analitico semplificato e il confronto dei risultati con l'ausilio di software di modellazione di sistemi fisici. 

